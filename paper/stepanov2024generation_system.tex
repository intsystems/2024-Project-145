\documentclass{article}
\usepackage{arxiv}

\usepackage[utf8]{inputenc}
\usepackage[english, russian]{babel}
\usepackage[T1]{fontenc}
\usepackage{url}
\usepackage{booktabs}
\usepackage{amsfonts}
\usepackage{nicefrac}
\usepackage{microtype}
\usepackage{lipsum}
\usepackage{graphicx}
\usepackage{natbib}
\usepackage{doi}



\title{Создание персонализированных генераций изображений}

\author{Степанов Илья Дмитриевич \\
	\texttt{iliatut94@gmail.com} \\
	%% examples of more authors
	\And
	Elias D.~Striatum \\
	Department of Electrical Engineering\\
	Mount-Sheikh University\\
	Santa Narimana, Levand \\
	\texttt{stariate@ee.mount-sheikh.edu} \\
	%% \AND
	%% Coauthor \\
	%% Affiliation \\
	%% Address \\
	%% \texttt{email} \\
	%% \And
	%% Coauthor \\
	%% Affiliation \\
	%% Address \\
	%% \texttt{email} \\
	%% \And
	%% Coauthor \\
	%% Affiliation \\
	%% Address \\
	%% \texttt{email} \\
}
\date{}

\renewcommand{\shorttitle}{\textit{arXiv} Template}

%%% Add PDF metadata to help others organize their library
%%% Once the PDF is generated, you can check the metadata with
%%% $ pdfinfo template.pdf
\hypersetup{
pdftitle={A template for the arxiv style},
pdfsubject={q-bio.NC, q-bio.QM},
pdfauthor={David S.~Hippocampus, Elias D.~Striatum},
pdfkeywords={First keyword, Second keyword, More},
}

\begin{document}
\maketitle

\begin{abstract}
        Проблема: В генеративных моделях наблюдается широкий спектр проблем, однако одной из наиболее актуальных является сложность создания высококачественных изображений конкретных людей с точностью, передающей их уникальную идентичность.

        Цель работы: Предлагается сфокусировать внимание на разработке моделей, способных генерировать изображения заданного человека в разнообразных вариациях и с высоким разрешением.

        Требуется: Обучить методы IP-Adapter, DreamBooth на модели Stable Diffusion. Сравнить эти 2 подхода.
\end{abstract}


\keywords{IP-Adapter \and DreamBooth \and Stable Diffusion}

\section{Introduction}
\end{document}
